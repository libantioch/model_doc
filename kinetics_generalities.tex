A chemical reaction is ``a bunch of molecules
turning into another bunch of molecules'', kinetics\footnote{%
from the greek $\kappa\iota\nu\eta\sigma\iota\varsigma$, ``kinesis'', movement, to move}
is about answering the question ``how fast?''.
Thus chemical kinetics is the mathematical model
to calculate the rate at which the molecules disappear and
appear.

\subsubsection{Going forward}

\Antioch's kinetics is based on the elementary step
hypothesis. It means that, as far as the kinetics is
concerned, every reaction is an elementary step:
the reactants get together and produce the products
immediatly. Mathematically, it means
the partial orders are the absolute
value of the stoichiometric coefficients (see next).

Let's consider a chemical reaction:
\begin{chemicalEquation}
\ce{\scoefabs[A] A + \scoefabs[B] B ->[\rcons] \scoefabs[C] C + \scoefabs[D] D}
\label{genericX}
\end{chemicalEquation}
with \rcons\ the rate constant.
We want to model the evolution of the system, that is we want to
characterize 
$\doverdt{\conc[A]}$,
$\doverdt{\conc[B]}$,
$\doverdt{\conc[C]}$,
$\doverdt{\conc[D]}$.
Using the kinetics theory, we have:
\begin{equation}
\frac{1}{\scoef[A]}\doverdt{[A]} = 
\frac{1}{\scoef[B]}\doverdt{[B]} = 
\frac{1}{\scoef[C]}\doverdt{[C]} = 
\frac{1}{\scoef[D]}\doverdt{[D]} = 
\rcons\conc[A]^{\scoefabs[A]}\conc[B]^{\scoefabs[B]}
\end{equation}
with \scoef[A]\ being the stoichiometric coefficient, which is defined by:\\[5pt]
$\left\{\begin{array}{ll}
\scoef[S] = \scoefabs[S] & \text{if \ce{S} is a product} \\
\scoef[S] = -\scoefabs[S] & \text{if \ce{S} is a reactant} \\
\end{array}\right.$\\[5pt]
So the game is to define the rate constant. 

A rate constant is characterized by two things:
\begin{itemize}
\item the kinetics model,
\item the chemical process.
\end{itemize}
The kinetics model will mathematically describe the rate constant's dependence with
the temperature, it is noted \kinMod\ in this manual, the chemical process will
possibly add a pressure dependency, it is noted \chemProc, with \conc[M]\
denoting the pressure dependence.
\Antioch\ propose six different kinetics models and five chemical processes.
The rate constant is characterized thus, for a choice of a chemical process and
a kinetics model:
\begin{equation}
\rateCons = \chemProc
\end{equation}

\subsubsection{Going backward}

Usually, a reaction will be reversible, which means, if we consider
that reaction~\ref{genericX} is reversible, we should note it:
\begin{chemicalEquation}
\ce{\scoefabs[A] A + \scoefabs[B] B <=>[\fwdratecons][\bkwdratecons] \scoefabs[C] C + \scoefabs[D] D}
\label{genericXrev}
\end{chemicalEquation}
with \fwdratecons\ the forward rate constant and \bkwdratecons\ the backward rate constant.
In a given physico-chemical environment, this reaction will eventually reach
steady state, characterized by an equilibrium constant \Eqconst. 
This equilibrium constant is given by
\begin{equation}
\Eqconst[r] = \frac{\fwdratecons[r]}{\bkwdratecons[r]}
\label{therm:K_kin}
\end{equation}
for a reaction $r$.
It is possible to estimate it from thermodynamics considerations, 
using the relation
\begin{equation}
\Eqconst[r] = \left(\frac{\pz}{\Rg \Temp}\right)^\gamma \exp\left(-\frac{\DGibbsZ[r](\Temp)}{\Rg \Temp}\right)
\label{therm:K_therm}
\end{equation}
The demonstrations are given in appendix~\ref{demo-eq_kin} and \ref{demo-eq_therm}. 
Thus the backward rate constant is therefore known given:
\begin{itemize}
\item the forward rate constant;
\item the thermodynamics of the molecules.
\end{itemize}

\subsection{Going nowhere: steady state, a.k.a equilibrium}
\label{phys:equilibrium}
\subsubsection{With kinetics}
A steady state is defined by
%
\begin{equation}
\forall\:s,\quad \doverdt{\conc[s]} = 0
\label{equilibrium:def}
\end{equation}
%
The system to be solved is of the form:
\begin{equation}
A\times x = b
\end{equation}
with $b$ the vector of \mdot, $A$ the matrixes of \doverdm[E]{\mdot[s]} for
the species (rows are $s$ and columns are $E$) and $x$ the vector of the solution \mass.
To close the system, we use the mass conservation equation and use a
species to ensure it:
\begin{equation}
\sum_s \mass[s] = \mathrm{mass_{tot}}
\label{mass_cons}
\end{equation}
with $\mathrm{mass_{tot}}$ being a constant, here the density of mass of the system.

So, for $N$ chemical species, we have the system:
\begin{equation}
\left[\begin{array}{cccc}
\doverdm[s_1]{\mdot[s_1]}     & \doverdm[s_2]{\mdot[s_1]}     & \cdots & \doverdm[s_N]{\mdot[s_1]} \\
\doverdm[s_1]{\mdot[s_2]}     & \doverdm[s_2]{\mdot[s_2]}     & \cdots & \doverdm[s_N]{\mdot[s_2]} \\
\vdots                        & \vdots                        & \vdots & \vdots                    \\
\doverdm[s_1]{\mdot[s_{N-1}]} & \doverdm[s_2]{\mdot[s_{N-1}]} & \cdots & \doverdm[s_N]{\mdot[s_{N-1}]}\\
1                             & 1                             & \cdots & 1\\
\end{array}\right]
\left[\begin{array}{c}
\Delta\mass[s_1]\\
\Delta\mass[s_2]\\
\vdots\\
\Delta\mass[s_N]\\
\end{array}\right]
=
\left[\begin{array}{c}
\mdot[s_1]\\
\mdot[s_2]\\
\mdot[s_1]\\
\vdots\\
\mdot[s_{N_1}]\\
\sum_{s=1}^N\mass[s] - \mathrm{mass_{tot}}
\end{array}\right]
\label{eq:matrixes}
\end{equation}
The total fixed mass is calculated thanks to the ideal gas state equation
(see section~\ref{relations})
\begin{equation}
\mathrm{mass_{tot}} = \Mm[\mathrm{mix}] \frac{P}{\Rg T}
\label{tot_mass}
\end{equation}
with \Mm[\mathrm{mix}] calculated as seen in section~\ref{relations}.
Thus an initial guess of \massfrac\ is necessary.
If you don't have any idea, let's consider the situation
\begin{chemicalEquation}
\ce{A + B ->[k_1] C + D ->[k_2] E + F}
\label{youpi}
\end{chemicalEquation}
we have
\begin{equation}
\doverdt{\conc[C]} = k_1\conc[A]\conc[B] - k_2\conc[C]\conc[D]
\end{equation}
therefore, a first approximation can be
\begin{equation}
\conc[C^{(\text{approx})}] = \frac{k_1\conc[A]\conc[B]}{k_2\conc[D]} = \frac{\mathrm{prod}}{\mathrm{loss}}
\end{equation}

This will be efficient in somewhat easy situations, meaning you're looking for the
steady state of minor species for instance.

\subsubsection{With thermodynamics}
A thermodynamic phase is at equilibrium for a minimized Gibbs energy (at \Temp, \Press\ constant), with
the relation
\begin{equation}
\dd\Gibbs = \Vol\dd\Press - \Entr\dd\Temp + \sum_s\chempot[s]\dd\Mol[s]
\end{equation}
and we have Euler's identity
\begin{equation}
\Gibbs = \sum_s \Mol[s]\chempot[s]
\label{Euler_id}
\end{equation}
with also,
\begin{equation}
\left(\doverdext[r]{\Gibbs[s]}\right)_{\Temp,\Press} = \scoef[s,r]\chempot[s]
\end{equation}
We note
\begin{equation}
\DGibbs_r = \sum_s \scoef[s,r]\chempot[s] \left[= \sum_s \left(\doverdext[r]{\Gibbs[s]}\right)_{\Temp,\Press}\right]
\end{equation}
Considering
\begin{equation}
\chempot_s = \doverdn[s]{\Gibbs[s]}
\end{equation}
We have, deduced from~\ref{Euler_id}
\begin{equation}
\chempot[s] = \gibbs[s]
\end{equation}
Thus,
\begin{equation}
\chempot[s] = \chempotZ[s] + \Rg\Temp\ln\left(\frac{\press[s]}{\pz}\right)
     \left[ = \chempotZ[s] + \Rg\Temp\ln\left(\frac{\Press}{\pz}\right) + \Rg\Temp\ln\left(\molarfrac[s]\right) \right]
\end{equation}
The story behind chemical extent is:
\begin{equation}
\Mol[s] = \Mol[s](t=0) + \sum_r \scoef[s,r] \ext[r]
\end{equation}
Using the ideal gas equation:
\begin{equation}
\Press = \conc \Rg \Temp
\end{equation}
thus
\begin{equation}
\begin{split}
\chempot[s] & = \chempotZ[s] + \Rg\Temp\ln\left(\frac{\Rg \Temp \molar[s]}{\pz}\right) \\
            & = \chempotZ[s] + \Rg\Temp\ln\left(\frac{\Rg \Temp}{\pz}\left(\molar[s](t=0) + \sum_r \scoef[s,r]\frac{\ext[r]}{\Vol}\right)\right) \\
\end{split}
\end{equation}
and therefore
\begin{equation}
\begin{split}
\doverdext[r]{\chempot[s]}      & = \frac{\pz}{\Vol}\frac{\scoef[s,r]}{\molar[s](t=0) + \sum_{r'} \scoef[s,r']\frac{\ext[r']}{\Vol}}\\
\ddoverddext{\chempot[s]}{r}{l} & = -\frac{\pz}{\Vol^2}\frac{\scoef[s,l]\scoef[s,r]}{\left(\molar[s](t=0) + \sum_{r'} \scoef[s,r']\frac{\ext[r']}{\Vol}\right)^2}\\
\end{split}
\end{equation}
Usually, it's better to consider the system per unit of volume, 
using an extent of reaction per volume, thus the equations
become:
\begin{equation}
\begin{split}
\chempot[s]                     & = \chempotZ[s] + \Rg\Temp\ln\left(\frac{\Rg \Temp}{\pz}\left(\molar[s](t=0) + \sum_r \scoef[s,r]\ext[r]\right)\right) \\
\doverdext[r]{\chempot[s]}      & = \pz\frac{\scoef[s,r]}{\molar[s](t=0) + \sum_{r'} \scoef[s,r']\ext[r']}\\
\ddoverddext{\chempot[s]}{r}{l} & = -\pz\frac{\scoef[s,l]\scoef[s,r]}{\left(\molar[s](t=0) + \sum_{r'} \scoef[s,r']\ext[r']\right)^2}\\
\end{split}
\end{equation}

Equilibrium is defined by 
\begin{equation}
\forall\; r,\; \DGibbs_r = 0
\end{equation} 
or 
\begin{equation}
\min \Gibbs(\{\chempot[s]\}_s)
\end{equation}

\subsubsection{\texorpdfstring{$\forall\;r,\;\DGibbs_r = 0$}{Reaction enthalpy}}

\begin{equation}
\begin{split}
\DGibbs_r & = \sum_s \scoef[s,r] \chempot[s]\\
\doverdext[i]{\DGibbs_r}   & = \sum_s\scoef[s,r]\doverdext[i]{\chempot[s]} \\
\ddoverddext{\DGibbs_r}{i}{j} & = \sum_s\scoef[s,r]\ddoverddext{\chempot[s]}{i}{j} \\
\end{split}
\end{equation}

For $R$ reactions:
\begin{equation}
\left[\begin{array}{cccc}
\sum_s\scoef[s,1]\doverdext[1]{\chempot[s]} & \sum_s\scoef[s,1]\doverdext[2]{\chempot[s]} & \cdots & \sum_s\scoef[s,1]\doverdext[R]{\chempot[s]}\\
\sum_s\scoef[s,2]\doverdext[1]{\chempot[s]} & \sum_s\scoef[s,2]\doverdext[2]{\chempot[s]} & \cdots & \sum_s\scoef[s,2]\doverdext[R]{\chempot[s]}\\
\vdots &\vdots &\vdots &\vdots \\
\sum_s\scoef[s,R]\doverdext[1]{\chempot[s]} & \sum_s\scoef[s,R]\doverdext[2]{\chempot[s]} & \cdots & \sum_s\scoef[s,R]\doverdext[R]{\chempot[s]}\\
\end{array}\right]
\times
\left[\begin{array}{c}
\Delta\ext[1] \\
\Delta\ext[2] \\
\vdots \\ 
\Delta\ext[R]
\end{array}\right]
=
\left[\begin{array}{c}
\sum_s \scoef[s,1] \chempot[s] \\ 
\sum_s \scoef[s,2] \chempot[s] \\ 
\vdots \\ 
\sum_s \scoef[s,R] \chempot[s]
\end{array}\right]
\end{equation}

\subsubsection{\texorpdfstring{$\min\Gibbs(\{\chempot[s]\}_s)$}{Phase enthalpy minimization}}

\begin{equation}
\begin{split}
\Gibbs & = \sum_s\Mol[s]\chempot[s]
         = \sum_s\left(\Mol[s]^0 + \sum_r\scoef[s,r]\ext[r]\right)\chempot[s] \\
\doverdext[r]{\Gibbs} & = \sum_s \left[\left(\Mol[s]^0 + \sum_{r'}\scoef[s,r']\ext[r']\right)\doverdext[r]{\chempot[s]}
                                + \scoef[s,r]\chempot[s]\right] \\
\ddoverddext{\Gibbs}{r}{l} & = \sum_s \left[\left(\Mol[s]^0 + \sum_{r'}\scoef[s,r']\ext[r']\right)\ddoverddext{\chempot[s]}{r}{l}
                                        + \scoef[s,l]\doverdext[r]{\chempot[s]}
                                        + \scoef[s,r]\doverdext[l]{\chempot[s]}\right] \\
\end{split}
\end{equation}
Considering these equations per unit of volume, one obtains:
\begin{equation}
\begin{split}
\Gibbs & = \sum_s\left(\molar[s]^0 + \sum_r\scoef[s,r]\ext[r]\right)\chempot[s] \\
\doverdext[r]{\Gibbs} & = \sum_s \left[\left(\molar[s]^0 + \sum_{r'}\scoef[s,r']\ext[r']\right)\doverdext[r]{\chempot[s]}
                                + \scoef[s,r]\chempot[s]\right] \\
\ddoverddext{\Gibbs}{r}{l} & = \sum_s \left[\left(\molar[s]^0 + \sum_{r'}\scoef[s,r']\ext[r']\right)\ddoverddext{\chempot[s]}{r}{l}
                                        + \scoef[s,l]\doverdext[r]{\chempot[s]}
                                        + \scoef[s,r]\doverdext[l]{\chempot[s]}\right] \\
\end{split}
\end{equation}


