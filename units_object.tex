\Antioch\ takes care of the units for every
parameter thanks to the \Units\ object. This object
enables powerful unit evaluation and conversion,
of a physical quantity as well as an equation.

A description of this object and its dependencies is
given in the \Doxygen\ documentation, thus only a
brief description and some examples will be given here.

The \Units\ objects projects an unit on a unit basis,
defined by the SI system plus radian for angles. There are
three different parts:
\begin{itemize}
\item the symbol of the unit,
\item the power array (basis projection),
\item the coefficient to convert the unit to the basis.
\end{itemize}

All is defined so as to provide an easy use should the user
want to play (or test/analyze) the dimension of some physical
quantity.

The basis is defined as: \unitbase.

\subsection{Simple example}

Let's consider an energy for instance, say parameter
$E$ is given in \unit{cal}. You would initialize a \Units\
object by 
\begin{verbatim}
Units<Scalar> E_unit("cal")
\end{verbatim}
\prog{Scalar} being the precision you want, typically \prog{float},
\prog{double} or \prog{long double}.

\subsection{Combining units}
\subsection{Restrictions}
No partial power! The physical senses of it is questionnable
anyway, \Antioch\ will not let you define in any way
a partial power.
\subsection{Goodies}
An object to automatically get the unit
from an equation:
\begin{verbatim}
define operators:
+ and - is check of homogeneity plus choosing a unit and
        calculating in chosen unit

* is adding unit and performing calculations

/ is substracting unit and performing calculations

pow(a,b) is multiplying `a' unit power by `b',
        needs check to verify that
    - the power is unitless
    - the resulting unit will not have a non integer unit

log requires unitless

cos/sin/... requires angles
\end{verbatim}
